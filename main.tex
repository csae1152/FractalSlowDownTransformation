\documentclass[a4paper]{article}

%% Language and font encodings
\usepackage[english]{babel}
\usepackage[utf8x]{inputenc}
\usepackage[T1]{fontenc}

%% Sets page size and margins
\usepackage[a4paper,top=3cm,bottom=2cm,left=3cm,right=3cm,marginparwidth=1.75cm]{geometry}

%% Useful packages
\usepackage{amsmath}
\usepackage{graphicx}
\usepackage[colorinlistoftodos]{todonotes}
\usepackage[colorlinks=true, allcolors=blue]{hyperref}

\title{Fractal Slow Down Transformation - FSDT}
\author{Helmut Steiner}

\begin{document}
\maketitle

\begin{abstract}
The present transformation aims to study the Triangular Intuitionistic Fuzzy Multiple Criteria Decision Making (TIF-MCDM)
problem for finding the best option where the phonetic factors for the criteria are pre-characterized. In perspective of this, a new
parametric transformation based distance measures has been proposed and implemented for various conceivable estimations of parameters. 
\end{abstract}

\section{Introduction}

Metric spaces such as Euclidean space, the sphere, and projective space possess rich families of simple geometrical transformations $f:X→X$. Examples are affine transformations of Euclidean space, Möbius and quadratic conformal mappings on the sphere, and projective transformations on projective space. The space and the mappings are simple to describe explicitly, and they are smooth.

Since the influences of subjective and objective factors, it is not easy for decision makers to 
give the accurate evaluation on complex things in practical decision problems. There
usually exist some hesitations for decision makers to assess the fuzzy and uncertain 
quantities. Therefore, Atanassov[6] introduced the concept of intuitionistic fuzzy set (IFS) 
characterized by a membership function and a non-membership function, which is a 
generalization of the concept of fuzzy set.

However, the domain of intuitionistic fuzzy set and interval-valued intuitionistic fuzzy
set are discrete sets, therefore they are only used to indicate the extent to which the criterion 
does or does not belong to some fuzzy concepts[25]. To overcome this flaw, Shu et al.[22]
gave the definition and operational laws of triangular intuitionistic fuzzy number. A 
prominent characteristic of the triangular intuitionistic fuzzy set is that its domain is a 
consecutive set. Some authors had paid attention to the research on triangular intuitionistic
fuzzy numbers [16, 17, 20, 21, 26], these researches can be roughly classified into two 
types: decision making methods[11, 17, 18, 20, 26, 27] and aggregation operators, which 
are respectively reviewed as follows. In the aspect of decision making methods, Li 
[16]pointed out and corrected some errors in the definition of the operational laws of
triangular intuitionistic fuzzy numbers introduced by Shu et al.[22]. Li[17] discussed the 
concept of triangular intuitionistic fuzzy number and the ranking method of triangular 
intuitionistic fuzzy number on the basis of the concept of a ratio of the value index to the 
ambiguity index as well as applications to MADM problems. Nan et al. [20]defined the 
ranking order relations of triangular intuitionistic fuzzy number, which are applied to 
matrix games with payoffs of triangular intuitionistic fuzzy number. Wan[26]introduced the 
notions of possibility mean and variance for triangular intuitionistic fuzzy numbers, 
developed a new decision method based on possibility mean and variance of triangular 
intuitionistic fuzzy numbers. Li et al[18] defined the values and ambiguities of the 
membership degree and the non-membership degree for triangular intuitionistic fuzzy 
number as well as the value-index and ambiguity-index, and developed a ranking method 
based on value and ambiguity. In the aspect of aggregation operators[21, 28], Robinson P. 
J[21] defined the triangular intuitionistic fuzzy ordered weighted averaging (TIFOWA) 
operator and the triangular intuitionistic fuzzy hybrid aggregation (TIFHA) operator, and an 
extended TOPSIS model is developed to solve the multiple attribute group decision making 
problems under triangular intuitionistic fuzzy sets by using its correlation coefficient. 
Wang[28] proposed new arithmetic operations and logic operators for triangular 
intuitionistic fuzzy numbers and applied them to fault analysis of a printed circuit board 
assembly system. Through the existing literature, we can found that the aggregation 
operators of triangular intuitionistic fuzzy numbers is still quite limited, and the 
methods for ranking triangular intuitionistic fuzzy numbers are a bit complicated, 
which are inconvenient to compare triangular intuitionistic fuzzy numbers. 

Compared with intuitionistic fuzzy numbers(IFNs), triangular intuitionistic fuzzy 
numbers are defined by using triangular fuzzy numbers expressing their membership and 
non-membership functions, which makes the membership degrees and the non-membership 
degrees no longer relative to a fuzzy concept “Excellent” or “Good”, but relative to the 
triangular fuzzy number; then, the information of decision makers can be reflected exactly 
and can be expressed in different dimensions[30]. Thus, the information of decision makers 
can be reflected exactly and can be expressed in different dimensions than IFNs. As the 
aggregation operators are critically important tools of information fusion in multiplecriteria decision making (MCDM) problems. The aim of this paper is to present some 
aggregation operators of triangular intuitionistic fuzzy numbers and a new method for 
ranking triangular intuitionistic fuzzy numbers. Thereby, a new multi-criteria decision 
making method using triangular intuitionistic fuzzy number is then proposed based on 
geometric average operators of triangular intuitionistic fuzzy numbers. 

In order to do that, this work is set out as follows. In Section 2, some basic concepts and 
operation laws related to triangular intuitionistic fuzzy numbers are introduced, and the 
distance of triangular intuitionistic fuzzy number is defined. In Section 3, the expected 
values, the score function and the accuracy function of triangular intuitionistic fuzzy 
number are given, and the ranking method is developed for triangular intuitionistic fuzzy 
numbers based on the score values and the accuracy function values. In Section 4, the 
concept of the triangular intuitionistic fuzzy weighted geometric (TIFWG) operator, the 
triangular intuitionistic fuzzy ordered weighted geometric (TIFOWG) operator and the 
triangular intuitionistic fuzzy hybrid geometric (TIFHG) operator are proposed and their 
desirable properties are studied. In Section 5, based on the ITFWG, TIFOWG and ITFHG 
operators, a new method to solve multi-criteria decision making problems with triangular 
intuitionistic fuzzy information is proposed. Finally, an illustrative example is given to 
verify the developed approach.


\section{Fractal Geometry}

A fractal drum is a bounded open subset of Rm with a fractal boundary. A difficult problem is to describe the relationship between the shape (geometry) of the drum and its sound (its spectrum). In this book, we restrict ourselves to the one-dimensional case of fractal strings, and their higherdimensional analogues, fractal sprays. We develop a theory of complex dimensions of fractal strings, and we study how these complex dimensions relate the geometry and the spectrum of fractal strings.

\subsection{Local structure of fractals}
Classical calculus involves finding local approximation to curves and surfaces by tangent lines and planes.

\subsection{Fractal decomposing algorithm}

Let $X_1, X_2, \ldots, X_n$ be a sequence of independent and identically distributed random variables with $\text{E}[X_i] = \mu$ and $\text{Var}[X_i] = \sigma^2 < \infty$, and let
\[S_n = \frac{X_1 + X_2 + \cdots + X_n}{n}
      = \frac{1}{n}\sum_{i}^{n} X_i\]
denote their mean. Then as $n$ approaches infinity, the random variables $\sqrt{n}(S_n - \mu)$ converge in distribution to a normal $\mathcal{N}(0, \sigma^2)$.

\subsection{Fractal transformations}
\subsection{Tropical Motivic Integration}

\section{Fractal Brain Theory}
\section{Reularity Structure of SPDE}


\bibliographystyle{alpha}
\bibliography{sample}

\end{document}